\documentclass[11pt]{article}%
\usepackage[utf8]{inputenc}

\usepackage{amsfonts}
\usepackage{fancyhdr}
\usepackage{comment}
\usepackage[a4paper, top=2.5cm, bottom=2.5cm, left=2.2cm, right=2.2cm]%
{geometry}
\usepackage{times}
\usepackage{amsmath}
\usepackage{changepage}
\usepackage{amssymb}
\usepackage{graphicx}
\usepackage{float}
\usepackage{hyperref}
\usepackage{wrapfig}
\setcounter{MaxMatrixCols}{30}
\newenvironment{proof}[1][Proof]{\textbf{#1.} }{\ \rule{0.5em}{0.5em}}

\newcommand{\Q}{\mathbb{Q}}
\newcommand{\R}{\mathbb{R}}
\newcommand{\C}{\mathbb{C}}
\newcommand{\Z}{\mathbb{Z}}

\begin{document}

\title{Information Visualization : Mobility in Brussels}
\author{Bruno M. Cabral, Pierre Gerard, Titouan Christophe, Luiz N. Junior}
\date{Vrije Universiteit Brussel}
\maketitle


\section{Introduction}
The mobility and public transportation in Brussels has a great importance in daily life of people using it.  Brussels region public transportation is organized into a vast number of lines that are divided into metro, tramways and buses, so the system is quite susceptible to suffer with exceptional events that happen in the city, which many times cause delays on its lines schedules, consequently, affecting its users.

The main goal of the project is trying to answer two questions which are related to this context of mobility via information visualization:

\begin{itemize}
	\item Is it possible to visualize the impact of street works or special events on brussels public transports ?
	\item Are there areas in the city that are significantly less served by public transports ?
\end{itemize}


\section{Data sources}
\subsection{Public Transportation Stops}
The public transportation network on Brussels region is managed by Brussels Intercommunal Transport Company (STIB/MIVB). The dataset used providing all stops locations (coordinates) and names is available at Open Data Brussels web portal\footnote{\url{http://opendata.bruxelles.be/explore/dataset/arrets-stib/}}.

\subsection{Lines Itineraries}
Informations about the stib lines were scrapped from the STIB mobile application\footnote{\url{http://m.stib.be/}}. From his website we got the following informations:
\begin{itemize}
    \item The itinerary, that is the sequence of stops a line pass by from a terminus to another
    \item The color of the line
    \item Wether it is a tramway, a metro or a bus line
\end{itemize}

\subsection{Vehicles flow}
\begin{figure}[H]
    \center
    \includegraphics[width=.3\textwidth]{stibmobile.png}
    \caption{\label{fig:stibmobile-app} The well-known frontend for the STIB API we used}
\end{figure}

In order to get the travel time for each line, a script is periodically querying the SITB mobile application API. We obtain a snapshot of the vehicles position every 20 seconds for each line. It is not possible to have a better resolution without trouble for the STIB servers. These snapshots are then processed to obtain the travel times. At this step, we have a set of tuples \texttt{(line, from\_stop\_id, to\_stop\_id, departure, arrival)}, stored in a relational database (PostgreSQL). As the amount of data is quite large, we rely on database indexes and aggregation functions to query for data.

\subsection{Delay events}
Some specifics period events that happened in the city in the last months were added manually to the application, i.e. The european summit that happened in March.                       
                       
\section{Architecture}
Our visualization is structured as a single page web application, written in ES6, and compiled into javascript using babeljs\footnote{\url{http://babeljs.io/}}.

We use Leaflet\footnote{\url{http://leafletjs.com/}} as map engine, the Ion Range Slider\footnote{\url{http://ionden.com/a/plugins/ion.rangeSlider/en.html}} for datetime pickers, and jQuery for DOM manipulation and asynchronous requests. Access to the database is provided by an API written in Python with Flask\footnote{}

\section{The visualization}

\subsection{Modal}

To ease the way for users unfamiliar with the visualization, we created a modal. That is a documentation page overlaying the visualization itself providing guidance to the user. That idea come from other visualization found on the web (eg : World Atlas of Language Structures \cite{visuawithmodal}).

\subsection{Map view}

\subsubsection{Stop density}


\subsubsection{Lines}

\subsection{Detailed view}

% Wrapping a figure does not want to work

%\begin{wrapfigure}{l}{0.35\textwidth}
%\includegraphics[width=0.95\linewidth]{images/tooltip.png} 
%\label{fig:subim1}
%\end{wrapfigure}

%Clicking on a line open a tooltip offering detailed information about the line.

%\textbf{Ah ha!} On the image on the left, we can this that between Trone and Science, the 22 and 34 bus are going faster than the other

%\vspace{6 cm}

\subsection{Events and time selection}

\section{Acknowledgement}

We would like to thanks Nikita Marchant and it's STIB data project \cite{nikita} project which enables us to get the delay data in the form of a postgresql relational database.

\section{Conclusion}


 
\begin{thebibliography}{9}
%\bibitem{nikita} 
%Michel Goossens, Frank Mittelbach, and Alexander Samarin. 
%\textit{The \LaTeX\ Companion}. 
%Addison-Wesley, Reading, Massachusetts, 1993.
 
\bibitem{nikita} 
Nikita Marchant: "Estimation en temps réel des temps de trajet dans un réseau de bus à l'aide de données historiques"
\\\url{https://github.com/C4ptainCrunch/info-f-308}

\bibitem{visuawithmodal} 
World Atlas of Language Structures,
\\\url{http://www.puffpuffproject.com/languages.html}



\end{thebibliography}

\end{document}